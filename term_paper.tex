\documentclass[
    a4paper,
    stu,
    12pt,
   noextraspace,
   floatsintext,
  %  draftall
]{APA7}

\usepackage{lipsum}

\usepackage[american]{babel}

\usepackage{csquotes}
\usepackage[style=apa,backend=biber]{biblatex}
\addbibresource{term_paper.bib}
% \addbibresource{term.bibtex}

\title{Meaningful Ideas through Ample Research and Information Literacy}
% \shorttitle{Sample Document}

\authorsnames{Samuel k. Njenga,Elon Beti,Umutesi Christella,Nabila Benshaban,Isacko Adele,Mercy Wambugu}
\authorsaffiliations{USIU University}
\course{ENG 1106E: Composition 1}
\professor{Dr. Naom Nyarigoti}
\duedate{16th November 2022}

\begin{document}
\maketitle
The task of the academic librarian is constantly in a state of nonstop modification and requires rigidity to an ever-adding number of changes. Additionally, academia is enduring new forms of technology, and librarians are needed to understand the arising technology. Librarians are assigned to tutor scholars' best research strategies while also measuring their literacy towards new technology. The library has observed significant technological advances within the last two decades. With a boost in several different online source options, scholars face the challenge of finding a proper starting position when researching. The institution's non-traditional scholars are an individual group that may be disadvantaged. The last time multiple of these scholars utilized a library, a physical hunt of a card catalog was necessary for finding a specific book, or they would count on a serial index to point them in the direction of a particular scholarly essay. These scholars are not only utilizing a computer but are utilizing a said computer to explore through several databases and across millions of essays for the best accessible source.

Education is an active process where scholars need to get enthralled in the assignment, partake in the lecture and break down the way conceptions have been created. Scholars must not only sit and hear what instructors explain; they need to write, ask questions and get involved in class conversations \parencite[]{ferlazzo_2021}.
The visually oriented students have trouble listening to a teacher in front of the class for two hours. However, they are eager to learn in collaborative environments where they must work in groups rather than listen to a teacher \parencite[]{francescato2006evaluation}.

A better literacy achievement can be attained when scholars get further involved in their education when they undergo and understand the content more. The more active scholars are, the quicker they learn and improve their skills. Interactive elements can assist in enhancing this process. A game is a way of presenting interactivity, but there are others. Visual elements, for instance, capture the attention of the Net Generation better than textbook-based factors. Active literacy can be applied in other settings in a web-based tutorial. When a web-based tutorial is enhanced with other interactive elements plushly imaged, it might gain the same results as a game. The clue is to engage scholars in their education, to get them to actively share and understand the content within the short period of occasion libraries have to educate information learning \parencite[]{van2010serious}.

The nature of information intake, assimilation, and circulation have significantly changed as society transitions into the digital age. The overwhelming volume and quantity of information we are exposed to today has changed how people perceive, share, and use it. People have responded to the rapidly developing digital world and the massive amounts of unregulated data it contains by filtering information into forms that are easier to understand and more relatable and consistent with their worldviews.

A so-called "Post-truth" society has emerged due to the excessive information intake fueled by the internet, where individuals choose to accept information that supports their pre-existing ideologies and views rather than attempting the challenging work of determining the truth. As the online environment grows increasingly challenging and complex, uncertainty keeps growing as people prioritize familiar information over trustworthy ones \parencite{DEPAOR2020102218}. Students need to  Check the author and source's qualifications, evaluate the sources the author cites, and look at the publication date to see if a website is reliable. Determining the publisher's credibility and looking into the evaluations and recommendations the source has received can also benefit the students.

The internet has made it much easier for students to find information, whether it be through Google searches, YouTube videos, or even online libraries. These tools provide students with rapid and portable information sources, thanks to this practical informational technique, which offers them access to a wide variety of knowledge and allows them to access it from anywhere. Students can finish their projects at any time, even while traveling.  
According to \Textcite{Judson2010}, there are two theories by which technological information literacy helps students; Technology literacy boosts self-confidence, promoting better academic results (confidence theory). Technology literacy refers to an improved capacity to use technological devices as knowledge mediators (mediation theory )

Students must be able to find, assess, and use the information correctly and ethically to be information literate \parencite{loveys2014development}. A few ways students can achieve this is by educating themselves on how to avoid it, and the ethics surrounding it could help enhance information literacy. Additionally, using different types of sources while doing research, such as library books, newspapers, or web articles, could help students to grow their information literacy. According to the article by \Textcite[]{nyarigoti2020assessment}, students could also seek assistance from librarians when finding a book to find the resources needed. Furthermore, including information literacy in the curriculum could also aid in strengthening information literacy by educating students on the Dos and not. Students should also seek to keep a search journal which helps keep track of databases they have used and keywords that have produced effective outcomes.

The motivation behind this expository paper was to explore and recognize the dire need for students to embrace various research tools to nurture a well-informed generation. There is no doubt that in today's generation lies a society that is bedeviling a multitude of challenges that cut across health, agriculture, security, and many other sectors, whose solutions squarely rely on meaningful ideas generated through ample research and information literacy. It is recommended that the government, universities, and other stakeholders in the learning sector should strive to avail financial, workforce, and moral support. The support will go a long way in providing a primary avenue for research and information literacy to prevail in the interest of today and future generations.

\printbibliography
\end{document}